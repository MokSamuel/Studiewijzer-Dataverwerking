\chapter{Studiewijzer}

Dit dictaat bevat informatie voor het vak "Dataverwerking" wat wordt gegeven in het 2e en 3e kwartiel van het 2e jaar van de opleiding Chemie. Dit dictaat begint met een overzicht van dit onderdeel, het materiaal, de leerdoelen, en de vorm van toetsing en beoordeling. Vervolgens bevat dit document achtergrondinformatie en materiaal wat nodig is tijdens het uitvoeren van dit vak.


\section{Materiaal}
Het materiaal wat je nodig hebt voor dit vak is beschikbaar op Blackboard in de module Chemical Fingerprinting. Hier vind je o.a. databases, voorbeeldbestanden, etcetera. 

Naast het materiaal wat er wordt aangeboden is er geen boek nodig voor dit vak. Wel heb je een laptop nodig met software. De software die wordt gebruikt in dit vak (installeer in ieder geval de eerste 3 voordat de lessen beginnen):
\begin{enumerate}
    \item Microsoft Office (Excel, Onedrive, Access): Downloaden op \href{https://office.com}{\textsf{office.com}}. Als je inlogt met je Saxion-account kan je gratis het complete office-pakket downloaden. Deze software wordt gebruikt vanaf week 3. 
    \item Blue Sky Statistics: Downloaden vanaf \href{https://www.blueskystatistics.com/}{\textsf{blueskystatistics.com}}. Dit softwarepakket is gratis te downloaden en wordt gebruikt vanaf week 1. 
    \item RStudio: Downloaden vanaf \href{https://rstudio.com/}{\textsf{rstudio.com}}. Dit softwarepakket is gratis te downloaden en wordt gebruikt vanaf week 7.
    \item Python: Dit wordt gebruikt in kwartiel 3, maar kan al goed van pas komen dit kwartiel! Je kan dit dus alvast installeren. Om dit (gratis) te downloaden kan je terecht op de Anaconda website \href{https://www.anaconda.com/distribution/}{anaconda.com}.
    \item (optioneel) SPSS: Dit is in leerjaar 1 gebruikt en kan worden gebruikt voor bepaalde analyses. Dit wordt echter niet aangeraden, en er is geen ondersteuning voor deze software in dit vak. SPSS kan je aanschaffen op \href{https://www.surfspot.nl/}{surfspot.nl}.
\end{enumerate}

\section{Leerdoelen}
De leerdoelen voor dit vak zijn algemeen geformuleerd en worden tijdens de lessen verder toegelicht.
\begin{itemize}
    \item Het kunnen uitvoeren en correct interpreteren van een multivariate clusteranalyse
    \item Het kunnen uitvoeren en correct interpreteren van een principale componentenanalyse
    \item Gegevens systematisch en gestructureerd op kunnen slaan aan de hand van een simpel datamanagementplan (DMP)
    \item Het begrijpen en kunnen toepassen van metadatadocumenten bij de opslag van data
    \item Het bewerken van verkregen gegevens voor langdurige opslag en mogelijke verdere verwerking
    \item Kennis verkrijgen over de werking van databases
    \item Het kunnen werken met een eenvoudige acces-database
    \item Het kunnen opzetten van een 2\textsuperscript{n} factorial design
    \item De basisprincipes van design of experiments begrijpen en kunnen toepassen
\end{itemize}

\section{Planning}

Iedere week komt er een ander onderwerp aan bod. Bij ieder onderwerp hoort een opdracht. Voor het overzicht is onderstaande tabel gemaakt:

\begin{table}[h]
\begin{tabular}{|l|l|l|}
\hline
\rowcolor[HTML]{009C82} 
Weeknummer & Onderwerp                    & Opdracht                         \\ \hline
1          & Clusteranalyse               &                                  \\ \hline
2          & Principale Compentenanalyse  & Verwerken in project kwartiel 2 \\ \hline
\rowcolor[HTML]{C0C0C0} 
3          & Datamanagementplan           & Opdracht DMP case                                  \\ \hline
\rowcolor[HTML]{C0C0C0} 
4          & Importeren \& verwerken data & Opdracht DMP project             \\ \hline
5          & Databases 1                  &                                  \\ \hline
6          & Databases 2                  & Opdracht databases               \\ \hline
\rowcolor[HTML]{C0C0C0} 
7          & Design of Experiments 1      &                                  \\ \hline
\rowcolor[HTML]{C0C0C0} 
8          & Design of Experiments 2      & Verwerken in project kwartiel 3 \\ \hline
\end{tabular}
\end{table}


\section{Toetsing}

Dit vak wordt in meerdere onderdelen getoetst. Een deel van het vak wordt getoetst in kwartiel 3 bij de eindopdracht dataverwerking (3 EC). De rest wordt getoets bij de praktijkopdrachten van kwartiel 2 en 3: bij het project chemical fingerprinting, urban mining, en met losse opdrachten bij professional skills 2. De details zijn te vinden op de Blackboardpagina's van de relevante kwartielen. 